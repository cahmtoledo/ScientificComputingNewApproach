\documentclass{article}
\usepackage{amsmath}
\usepackage[utf8]{inputenc}
\usepackage{indentfirst}

\title{Calculando a precisão da medida}
\author{Carmen Melo Toledo}
\date{January 2020}


\begin{document}

\maketitle

\section{Introdução}

Para achar o erro aproximado vamos aproximar a função de uma gaussiana dada pela equação

\begin{equation}
    g(x) = \frac{1}{\sigma\sqrt{2\pi}}\exp{\Big( - \frac{(x-x')}{2\sigma} \Big)}
\end{equation}

\subsection{Aproximando por uma gaussiana}

seja $L=log(P(X|\{D\}, \theta))$

se fizermos a expansão de taylor até o segundo termo teremos:

\begin{equation*}
    L(x) \approx L(x') + \frac{1}{2} \frac{\partial^2 L}{\partial x^2}\Big|_{x'} (x-x')^2
\end{equation*}

De onde temos que 

\begin{equation*}
P(x| \{ D \}, \theta) \approx C \exp{\Big( \frac{1}{2} \frac{\partial^2 L}{\partial x^2}\Big|_{x'} (x-x')^2  \Big)}
\end{equation*}



\subsection{Calculando $\sigma$}

Se compararmos com a equação da gaussiana percebemos que

\begin{equation}
    \sigma = \Big(- \frac{\partial^2 L}{\partial x^2} \Big)^{-\frac{1}{2}}
\end{equation}


\subsection{$\sigma$ na medida com erro}

Na parte dois vimos que

\begin{equation*}
    \frac{\partial L}{\partial \mu} = \frac{\sum_{k=1}^{N}(D_k-\mu)}{\sigma'^2} = 0 \iff \mu = \frac{1}{N}\sum_{k=1}^{N}(D_k)
\end{equation*}

Performando a segunda derivada chegamos que

\begin{equation*}
    \frac{\partial^2 L}{\partial \mu^2} = \frac{\sum_{k=1}^{N} 1}{\sigma'^2} = \frac{N}{\sigma'^2} 
\end{equation*}

de onde tiramos que $\sigma = \frac{\sigma'}{\sqrt{N}}$

\subsection{$\sigma$ na medida sem erro}

Nesse caso teremos
\begin{equation*}
      \frac{\partial^2 L}{\partial \mu^2} = -\frac{N(N-1)}{\sum(x_k-\mu)^2}
\end{equation*}

De onde sai que

\begin{equation*}
    \sigma = \sqrt{(N-1)^{-1}\sum(x_k-\mu)^2} / \sqrt{N}
\end{equation*}



\end{document}
