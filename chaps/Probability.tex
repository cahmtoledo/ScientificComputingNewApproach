\documentclass{book}
\usepackage[utf8]{inputenc}
\usepackage[shortlabels]{enumitem}
\usepackage{mathtools}


\begin{document}

\chapter{\emph{Probability}}
%\chapter{\emph{Probability}} 

\label{probability}

\section{What is probability?}
Kolmogorov axioms

\section{What is it, really?}

\section{Random variables}

\section{Independence and Bayes theorem}

\subsection{Solved Exercises}

%\title{ExercisePregnancy}
%\author{Carmen Melo Toledo}


\subsubsection{Pregnancy Test}

    According to CDC (Center for Disease Control and Prevetion):
    

    \begin{center}
     \begin{tabular}{||c | c||} 
     \hline
     Contraceptive Method & Percentage of failure\footnote{percentage of undesired pregnancy after a year} \\ [0.9ex] 
     \hline\hline
     Male Codon & 18\% \\ 
    \hline
    Pill & 9\% \\ 
    \hline
    Fertility-Awareness based method & 24\% \\ 
    \hline
    IUD & 0.5\% \\
    \hline
    \end{tabular}
    \end{center}
    consider a pregnancy test that 99\% accurate to positive and to negative\footnote{pregnancy tests usually are less accurate to avoid fake negatives, but to exemplify let's assume it}  and estimated the percentage of having a typical fertile couple have a fake positive after a year using:
    \begin{enumerate}[(a)]
        \item Male Codon
        \item Pill
        \item IUD
        \item Pill and Male Codon
    \end{enumerate}
    


\textbf{Answer}
To solve this problem we shall use tow equations that we saw earlier
\large{Bayes: }
\begin{equation*}
 \Pr[a|b]=\frac{\Pr[b|a]*\Pr[a]}{\Pr[b]}
\end{equation*}


 \large{theorem of total probability:}
\begin{equation*}
    \\ \Pr[a]=\Pr[a|b]*Pr[b]+Pr[a|\sim b]*Pr[\sim b]
\end{equation*}

So in general we have:

\begin{equation*}
    \large{\Pr[not pregnant| positive]=\frac{\Pr[positive|not pregnant]*\Pr[not pregnant]}{\Pr[positive]}}
\end{equation*}

where $\Pr[positive|not pregnant]$  is  the  accuracy,

$\Pr[not pregnant]$ is 100\% - chance of failure and 

$\Pr[positive]$ will be given by the second theorem

\begin{equation*}
\large{\Pr[positive]=\Pr[Positive|not pregnant}*\Pr[not pregnant]+\Pr[Positive|pregnant]*\Pr[pregnant]    
\end{equation*}

Let's apply to each letter:

a)

$\Pr[positive|not pregnant] = 1\%$

$\Pr[not pregnant]= (100-18)\%=82\%$

$\Pr[positive]=0.01*0.82+0.99*0.18=0.1864$

So we have

$\Pr[fake positive] = \frac{0.01*0.82}{0.1864}=0.044 or 4.4\%$


b)

$\Pr[positive|not pregnant] = 1\%$

$\Pr[not pregnant]= (100-9)\%=91\%$

$\Pr[positive]=0.01*0.91+0.99*0.09=0,0982$

So we have

$\Pr[fake positive] = \frac{0.01*0.91}{0.0982}=0.09266 or 9.3\%$

c)

$\Pr[positive|not pregnant] = 0.5\%$

$\Pr[not pregnant]= (100-0.5)\%=99.5\%$

$\Pr[positive]=0.01*0.995+0.99*0.005=0,0149$

So we have

$\Pr[fake positive] = \frac{0.01*0.995}{0.0149}=0.6678=66.8\%$

d)

Using codon and taking pill are independent events so the failure rate is the multiplication of the failure rates

Failure Rate: 0.0162 = 1.62\%

$\Pr[positive|not pregnant] = 1.62\%$

$\Pr[not pregnant]= (100-1.62)\%=98.38\%$

$\Pr[positive]=0.01*0.9838+0.99*0.0162=0,02588$

So we have

$\Pr[fake positive] = \frac{0.01*0.9838}{0.02588}=0.3801=38\%$

\subsubsection{exercise 2}


\section{Probability models}


\end{document}