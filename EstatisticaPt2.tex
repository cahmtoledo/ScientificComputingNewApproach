\documentclass{article}
\usepackage[utf8]{inputenc}
\usepackage{amsmath}
\usepackage{mathtools}


\title{Medindo com estatísticas}
\author{Carmen Melo Toledo}
\date{January 2020}

\begin{document}

\maketitle

\section{Introdução}
Suponha que queiramos medir um prego com uma régua com erro médio $\sigma$, como poderíamos fazer isso?

\subsection{Distribuição gaussiana}
É razoável supormos que a distribuição das medidas seguirá a curva gaussiana, ou seja

\begin{equation}
    P(M|\mu, \sigma) = (\sigma \sqrt{2\pi})^{-1} e^{\frac{-(M-\mu)^2}{2\sigma^2}}
\end{equation}

Onde M é a medida feita e $\mu$ é o valor de fato do prego que queremos descobrir.

Montando a equação de bayes temos

\begin{equation*}
    P(\mu| \{D\}, \sigma) = \frac{P(\{D\}|\mu, \sigma)P(\mu|\sigma)}{P(\{D\}|\sigma)}
\end{equation*}
É razoável considerarmos todos os valores de $\mu$ equiprováveis e vamos ignorar a evidência.

\begin{equation*}
    P(\mu|\{D\},\sigma) \propto P(\{D\}|\mu, \sigma)
\end{equation*}

Como os dados são independentes entre si

\begin{equation*}
    P(\{D\}| \mu, \sigma) = \prod_{k=1}^N P(D_k|\mu,\sigma)
\end{equation*}

\begin{equation*}
    P(\{D\}| \mu, \sigma) = (\sigma\sqrt{2\pi})^N \exp{\Big( \frac{-\sum_{k=1}^{N}(D_k-\mu)^2}{2\sigma^2}\Big)}
\end{equation*}

Jogando na equação e tirando o log dos dois lados chegamos em

\begin{equation*}
    L = log(\mu| \{D\} \sigma)) = C - \frac{\sum_{k=1}^{N}(D_k-\mu)^2}{2\sigma^2}
\end{equation*}

O valor mais provável de $\mu$ é aquele que zera a derivada de L com relação a $\mu$

\begin{equation*}
    \frac{\partial L}{\partial \mu} = \frac{\sum_{k=1}^{N}(D_k-\mu)}{\sigma^2} = 0 \iff \mu = \frac{1}{N}\sum_{k=1}^{N}(D_k)
\end{equation*}

Com isso chegamos a "surpreendente" conclusão que a medida mais provável é a média das medidas tomadas
\end{document}
